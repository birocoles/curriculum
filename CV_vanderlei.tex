%%%%%%%%%%%%%%%%%%%%%%%%%%%%%%%%%%%%%%%%%
% "ModernCV" CV and Cover Letter
% LaTeX Template
% Version 1.11 (19/6/14)
%
% This template has been downloaded from:
% http://www.LaTeXTemplates.com
%
% Original author:
% Xavier Danaux (xdanaux@gmail.com)
%
% License:
% CC BY-NC-SA 3.0 (http://creativecommons.org/licenses/by-nc-sa/3.0/)
%
% Important note:
% This template requires the moderncv.cls and .sty files to be in the same 
% directory as this .tex file. These files provide the resume style and themes 
% used for structuring the document.
%
%%%%%%%%%%%%%%%%%%%%%%%%%%%%%%%%%%%%%%%%%

%----------------------------------------------------------------------------------------
%	PACKAGES AND OTHER DOCUMENT CONFIGURATIONS
%----------------------------------------------------------------------------------------

\documentclass[10pt,a4paper,sans]{moderncv} % Font sizes: 10, 11, or 12; paper sizes: a4paper, letterpaper, a5paper, legalpaper, executivepaper or landscape; font families: sans or roman

\moderncvstyle{classic} % CV theme - options include: 'casual' (default), 'classic', 'oldstyle' and 'banking'
\moderncvcolor{blue} % CV color - options include: 'blue' (default), 'orange', 'green', 'red', 'purple', 'grey' and 'black'

\usepackage[colorlinks=true, urlcolor=cyan]{hyperref}

%\usepackage{lipsum} % Used for inserting dummy 'Lorem ipsum' text into the template

\usepackage[scale=0.8]{geometry} % Reduce document margins
\setlength{\hintscolumnwidth}{3.5cm} % Uncomment to change the width of the dates column
\setlength{\makecvtitlenamewidth}{5.5cm} % For the 'classic' style, uncomment to adjust the width of the space allocated to your name




%----------------------------------------------------------------------------------------
%	NAME AND CONTACT INFORMATION SECTION
%----------------------------------------------------------------------------------------

\firstname{Vanderlei} % Your first name
\familyname{C. Oliveira Jr.} % Your last name

% All information in this block is optional, comment out any lines you don't need
% \title{Researcher}
\address{\href{http://www.on.br/index.php/pt-br/}{Observat\'{o}rio Nacional - ON}}{Rio de Janeiro, Brazil}
% \mobile{(+55) 219 8154 4923}
\email{vanderlei@on.br; vandscoelho@gmail.com}

% The first argument is the url for the clickable link, the second argument is the url displayed in the template - this allows special characters to be displayed such as the tilde in this example
%\homepage{http://lattes.cnpq.br/7515548666449468}{lattes(portuguese)-florasolon}
\homepage{http://www.pinga-lab.org/people/oliveira-jr.html}

% \extrainfo{Nationality: Brazilian}

% The first bracket is the picture height, the second is the thickness of the frame around the picture (0pt for no frame)
\photo[80pt][0pt]{vanderlei.jpg}

%\quote{"A witty and playful quotation" - John Smith}

%----------------------------------------------------------------------------------------
\begin{document}
%\includegraphics[scale=0.30]{Aplication_form_ed.pdf}

\newpage
\makecvtitle % Print the CV title
\textcolor{gray}{\noindent\rule{18cm}{0.4pt}}

\textcolor{white}{force newline}\\

%----------------------------------------------------------------------------------------
%	ATUAL POSITION
%----------------------------------------------------------------------------------------
\section{Current Position}

\cventry{Sep 2013 -- present}{\textnormal{Associate Professor in Geophysics}}{Observat\'{o}rio Nacional - ON}{Rio de Janeiro, Brazil}{}{}{}s

% Arguments not required can be left empty

%----------------------------------------------------------------------------------------
%	RESEARCH INTERESTS
%----------------------------------------------------------------------------------------
\section{Research}

%\cventry{}{\textnormal{Geophysics; Potential fields (gravity and magnetic methods); data processing; modelling and inversion}}{}{}{}{}{}

%\textcolor{white}{force newline}\\

\cventry{}{\textnormal{After obtaining my undergraduate degree in Geophysics at the Instituto de Astronomia,}}{}{}{}% Geof\'{i}sica e Ci\^{e}ncias Atmosf\'{e}ricas da Universidade de S\~{a}o Paulo (IAG-USP), I received a master degree in Geophysics at Observat\'{o}rio Nacional, where I subsequently obtained a PhD degree. Nowadays I am a researcher at the department of Geophysics of Observat\'{o}rio Nacional. I am mainly interested in potential fields (gravity and magnetic methods), modelling and inversion.}}{}{}{}

\textcolor{white}{force newline}\\
% Arguments not required can be left empty
\textcolor{gray}{\noindent\rule{18cm}{0.4pt}}

%----------------------------------------------------------------------------------------
%	EDUCATION SECTION
%----------------------------------------------------------------------------------------
\textcolor{white}{force newline}\\

\section{Education}

\cventry{{Dec 2010 -- Jan 2013}}{Ph.D. in Geophysics}{Observat\'{o}rio Nacional - ON}{Rio de Janeiro, Brazil}{Thesis title (portuguese): "Processamento e inversão de dados de campos potenciais: Novas abordagens'', (english): "Processing and inversion of potential field data: New approaches''}{Advisor: Dr. Valeria C. F. Barbosa (Observat\'{o}rio Nacional, ON)} % Arguments not required can be left empty

\textcolor{white}{force newline}\\

\cventry{{Mar 2009 -- Nov 2010}}{Master in Geophysics}{Observat\'{o}rio Nacional - ON}{Rio de Janeiro, Brazil}{Thesis title (portuguese): "Inversão gravimétrica radial por camadas para a reconstrução de corpos geológicos 3D'', (english): "Radial gravity inversion for retrieving 3D geological bodies''}{Advisor: Dr. Valeria C. F. Barbosa (Observat\'{o}rio Nacional, ON)}  % Arguments not required can be left empty

\textcolor{white}{force newline}\\

\cventry{Mar 2004 -- Dec 2008}{Bachelor in Geophysics}{Universidade Federal Fluminense - UFF}{Rio de Janeiro, Brazil.}{Thesis Title: "Electromagnetic characterization of geological formations of S\~{a}o Francisco basin''}{Advisor: Dr. Sergio L. Fontes (Observat\'orio Nacional, ON).}  % Arguments not required can be left empty

\textcolor{white}{force newline}\\



%----------------------------------------------------------------------------------------
%	WORK EXPERIENCE SECTION
%----------------------------------------------------------------------------------------


%----------------------------------------------------------------------------------------
%	PUBLICATIONS 
%----------------------------------------------------------------------------------------


\section{Publications in refereed journals}

%\subsection{As First Author}

\begin{enumerate}

\href{http://pg.lyellcollection.org/content/early/2015/07/28/petgeo2013-013}{1. \textbf{Solon, F.F.}, Fontes, S.L., Meju, M. A. 2015. "Magnetotelluric imaging integrated with seismic, gravity, magnetic and well-log data for basement and carbonate reservoir mapping in the S\~{a}o Francisco Basin, Brazil'', Petroleum Geoscience Vol 21, No 4, November 2015, pp. 285 - 299, doi:10.1144/petgeo2013-013.}\\

\end{enumerate}
%\textcolor{white}{force newline}\\


\section{Submitted Publications}

\begin{enumerate}


\item \textbf{Solon, F.F.}, Fontes, S.L., La Terra, E.F., 2017. "Magnetotelluric evidence of crustal conductors in Parna\'{i}ba basin, NE Brazil". In revision for Geological Society of London (GSL) Special volume: Cratonic Basin Formation: A Case Study of the Parna\'{i}ba Basin of Brazil.\\


\end{enumerate}

%\textcolor{white}{force newline}\\

\section{Publications in Conference Proceedings}

\begin{enumerate}

\item \textbf{Solon, F. F.}; Fontes, S. L.; La Terra, E. F., 2016. "Electromagnetic studies in the Parna\'{i}ba Basin: structural characterization by MT imaging". In: 2016 AGU Fall Meeting, 2016, San Francisco, EUA - Oral presentation.\\

\item \textbf{Solon, F. F.}; Fontes, S. L. ; La Terra, E. F., 2015. "Electromagnetic studies in the Parna\'{i}ba Basin: structural characterization by MT imaging". In: 26th IUGG General Assembly, 2015, Praga, Czech Republic - Oral presentation.\\

\item \textbf{Solon, F. F.}; Gallardo, L. A.; Fontes, S. L., 2015. "Characterization of S\`{a}oFrancisco Basin, Brazil - Joint Inversion of MT, Gravity and Magnetic Data". In: 26th IUGG General Assembly, 2015, Praga, Czech Republic - Oral presentation. \\

\item \textbf{Solon, F. F.}; Fontes, S. L., 2014. "Electromagnetic studies in the Parna\'{i}ba Basin: structural characterization by MT imaging." In: 22nd EM Induction Workshop, 2014, Weimar, Germany - Poster presentation\\

\item \textbf{Solon, F. F.}; Tupinamba, M.; Miquelutti, L. G.; La Terra, E. F.; Fontes, S. L., 2013. "Ancient geological structures in the middle crust of southeast brazilian portion identified by geoelectrical results with Magnetellurics geoohysical methods". In: 13th International Congress of the Brazilian Geophysical Society, Rio de Janeiro, Brazil - Oral presentation.\\

\item La Terra, E. F.; Miquelutti, L. G.; Fontes, S. L.; \textbf{Solon, F. F.}; Pinto, V. R.; Braga, F.; Maciel, M. M.; Poenca, T.; Figueiredo, I., 2012. "Depth distribution of geological structures in western edge of Santos basin from integrated broad band magnetotellurics and geological mapping". In: 21th Electromagnetic Induction Workshop, 2012, Darwin, Australia.\\

\item \textbf{Solon, F. F.}; Fontes, S. L.; Flexor, J.M.; Meju, M.A., 2011. "Electromagnetic and seismic images from S\`{a}oFrancisco Basin Brazil: oil and gas perspectives?. In: SEG Technical Program Expanded Abstracts p. 650-654. San Antonio, EUA - Oral presentation.\\

\item \textbf{Solon, F. F.}; Fontes, S. L.; Flexor, J.M.; Meju, M.A., 2011. "Electromagnetic and Seismic characterization of onshore basement and carbonate structures from S\`{a}oFrancisco Basin, Brazil". In: 12th International Congress of the Brazilian Geophysical Society, Rio de Janeiro, Brazil - Oral presentation.\\

\item \textbf{Solon, F. F.}; Bijani, R.; Pinto, V. R.; Fontes, S. L.; Carrasquila, A. A. G., 2011. "Magnetotelluric investigation on the onshore Campos Basin". In: 12th International Congress of the Brazilian Geophysical Society, 2001, Rio de Janeiro, Brazil - Poster presentation.\\

\item \textbf{Solon, F. F.}; Fontes, S. L.; Flexor, J.M.; Meju, M.A., 2010. ": Electromagnetic Characterization of fractured basement and carbonate structures beneath thick overburden in Sao Francisco-Parnaiba Basins, Brazil". In: 20th Electromagnetic Induction Workshop, Giza Egypt - Poster presentation.\\

\item \textbf{Solon, F. F.}; Melgaco, P. P. P. S.; Fontes, S. L.; Meju, M. A., 2009. "Assinatura geoel\'{e}trica do Arco de S\~{a}o Francisco: encontro das bacias do S\~{a}o Francisco e Parna\'{i}ba". In: 11th International Congress of the Brazilian Geophysical Society, Salvador, Brazil - Poster presentation \\

\end{enumerate}

%\textcolor{white}{force newline}\\

\section{Participation in Research Projects}

\cventry{2013--2017}{\textnormal{Parna\'{i}ba Basin structure from Magnetotelluric (MT) imaging In Integrated geophysical studies in Parna\'{i}ba basin, Brazil: Parna\'{i}ba Basin Integrated Project -- PABIP, Observatorio Nacional/University of Oxford/BP Energy}}{Sergio L. Fontes (coordinator); Emanuele F. La Terra (co-participant); \textbf{Flora F. Solon} (co-participant)}{}{}{}

\cventry{2011--2013}{\textnormal{Characterization of S\~{a}oFrancisco basin, Brazil: joint inversion of multiple geophysical data}}{advisors: Sergio L. Fontes and Luis A. Gallardo}{}{}{}

\cventry{2009--2010}{\textnormal{Electromagnetic Characterization of fractured basement and carbonate structures beneath thick overburden in S\~{a}o Francisco-Parna\'{i}ba Basins, Brazil -- Observatorio Nacional}} {advisor: Sergio L. Fontes}{}{}{}

\cventry{2008--2009}{\textnormal{"Geoelectrical signature of the S\~{a}o Francisco high: boundary of S\~{a}o Francisco and Parna\'{i}ba basins" -- Observatorio Nacional}} {advisor: Sergio L. Fontes.}{}{}{}


%\textcolor{white}{force newline}\\


%----------------------------------------------------------------------------------------
%	AWARDS SECTION
%----------------------------------------------------------------------------------------
\section{Professional Experience}

\cventry{2011--2013}{\textnormal{Consultant in Geophysics at a multi-institutional project Subsalt Imaging at Observat\'{o}rio Nacional}}{Main Responsibilities: responsible for magnetotelluric survey and data processing}{}{}{}

\cventry{apr 2010--dec 2010}{\textnormal{Intership Program as geophysicist at Strataimage Consultoria Ltda}}{Worked with magnetotelluric survey, data processing, digitalization of seismic data}{}{}{}

\cventry{2008--2010}{\textnormal{Academic Trainee at Repsol YPF at the project: Morpho-tectonic analysis and stratigraphic significance of Structural Highs in Southeast Brazilian margin basins}}{Main Responsibilities: gravity and 2D seismic data interpretation of Campos and Esp\'{i}rito Santo basins, map and sedimentary analysis of Vitoria High}{}{}{}

\cventry{2009--2010}{\textnormal{Graduate Research Assistant at Observatorio Nacional}}{Main Responsibilities: Magnetotelluric Survey and processing: worked with Magnetotelluric survey and data processing at Rec\^{o}ncavo Basin using the ADU07 equipment of Metronix; magnetotelluric data processing and mapping fractured carbonates in S\~{a}o Francisco basin--Advisor: Sergio L. Fontes, Emanuele F. La Terra}{}{}{}


\section{Visiting Institutions}

\cventry{Oct 2012 -- Jan 2013}{Centro de Investigaci\'{o}n Cient\'{i}fica y de Educaci\'{o}n Superior de Ensenada -- CICESE}{Ensenada, BC}{ Mexico.}{collaboration with Dr. Luis Allonso Gallardo}{}{}{}





%----------------------------------------------------------------------------------------
%	COMMUNICATION SKILLS SECTION
%----------------------------------------------------------------------------------------

%\section{Communication Skills}

%\cvitem{2010}{Oral Presentation at the California Business Conference}
%\cvitem{2009}{Poster at the Annual Business Conference in Oregon}

%----------------------------------------------------------------------------------------
%	LANGUAGES SECTION
%----------------------------------------------------------------------------------------

\section{Languages}

\cvitemwithcomment{Portuguese}{\textnormal{Native language}}{}
\cvitemwithcomment{English}{\textnormal{Fluent}}{}
\cvitemwithcomment{French}{\textnormal{Basic - Basic words and phrases only}}{}


%----------------------------------------------------------------------------------------
%	COMPUTER SKILLS SECTION
%----------------------------------------------------------------------------------------

\section{Computing skills}

\cvitem{Operating systems}{Linux, Windows}
\cvitem{Programming}{Python, Matlab}


%----------------------------------------------------------------------------------------
%	REFERENCES
%----------------------------------------------------------------------------------------
\section{References}

\cventry{Dr. Sergio L. Fontes}{\textnormal{Senior professor at Observat\'orio Nacional (ON)}}{Rua Gen Jos\'e Cristino, S\~ao Crist\'ov\~ao, RJ, Brazil 20921-400}{}{}{sergio@on.br}{}

\cventry{Dr. Valeria Barbosa}{\textnormal{Senior professor at Observat\'orio Nacional (ON)}}{Rua Gen Jos\'e Cristino, S\~ao Crist\'ov\~ao, RJ, Brazil 20921-400}{}{}{barbosa@on.br}{}

\cventry{Dr. Emanuele F. La Terra}{\textnormal{Technologist at Observat\'orio Nacional (ON)}}{Rua Gen Jos\'e Cristino, S\~ao Crist\'ov\~ao, RJ, Brazil 20921-400}{}{}{laterra@on.br}{}

\cventry{Dr. Maxwell A. Meju}{\textnormal{Principal Geophysicist at PETRONAS}}{Kuala Lumpur, Malaysia}{}{}{maxwell\_ meju@petronas.com.my}{}




%----------------------------------------------------------------------------------------
%	COVER LETTER
%----------------------------------------------------------------------------------------

% To remove the cover letter, comment out this entire block

\clearpage

\textbf{Summary of Research Interests -- Flora F Solon}\\

I'm a geophysicist working mostly in applied geophysics, with emphasis on the magnetotelluric method. During my Ph.D. I worked mainly with data processing, modelling and inversion of MT data. I analysed several MT surveys in the S\~{a}o Francisco basin and in the Parna\'{i}ba basin, both intracratonic basins localized in north-east of Brazil. In S\~{a}o Francisco Basin, MT was used to locate crystalline basement and overlying carbonate rocks. For model appraisal, we analysed well log (gamma ray, deep resistivity and neutron porosity) data as well as seismic, gravity and magnetic profiles coincident with one MT line passing through the well. I participated in a multidisciplinary project carried out in the Parna\'{i}ba basin between Observatorio Nacional, University of Oxford and BP Energy of Brazil, under BP sponsored Parna\'{i}ba Basin Analysis Project (PBAP). MT data were acquired in an E-W profile in the Parna\'{i}ba basin in a profile of approximately 1430 km long, the first MT profile to cross the entire basin. I applied 3D modelling and inversion to image geoelectrical structures within the crustal basement. Conductive anomalies have been associated with the presence of interconnected graphites suggesting that the central Parna\'{i}ba region is located over former suture zones.

Recently I have tried to improve my computational skills working with modelling and inversion of MT data. This was possible with the use of open-source software like \href{http://www.sciencedirect.com/science/article/pii/S0098300414001794}{MTpy} [1], \href{http://www.simpeg.xyz/}{SimPEG} [2] and \href{http://www.fatiando.org/}{Fatiando a Terra} [3]. In addition, I developed small routines and programs using the Python language, which helped me during the development of the Ph.D. Among them, a routine to analyse the ambiguity of the MT method, a forward modelling and a 1D MT inversion code.


\subsection{Links}
[1] Krieger, L., Peacock, J. R. 2014, MTpy: A Python toolbox for magnetotellurics. Computers and Geosciences 72, 167--175.

[2] Cockett, Rowan, Seogi Kang, Lindsey J. Heagy, Adam Pidlisecky, and Douglas W. Oldenburg, 2015 SimPEG: An Open Source Framework for Simulation and Gradient Based Parameter Estimation in Geophysical Applications. Computers and Geosciences,  doi:10.1016/j.cageo.2015.09.015.

[3] Uieda, L., V. C. Oliveira Jr, and V. C. F. Barbosa (2013), Modeling the Earth with Fatiando a Terra, Proceedings of the 12th Python in Science Conference, pp. 91--98.



\end{document}




