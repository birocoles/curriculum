% list of papers

\nocite{*}
\bibliographystyle{cv}
%\bibliography{papers}
\begin{thebibliography}{10}

\bibitem{bastos_oliveirajr_2019}
B.~M.~S. Bastos, V.~C. Oliveira{ }Jr. Isostatic constraint for 2D non-linear
  gravity inversion on rifted margins. \emph{GEOPHYSICS} \textbf{2019},
  \emph{0}, ja.

\bibitem{araujo_hallsensor_2019}
J.~Araujo, A.~Reis, V.~C. Oliveira{ }Jr, A.~Santos, C.~Luz-Lima, E.~Yokoyama,
  L.~Mendoza, J.~Pereira, A.~Bruno. Characterizing Complex Mineral Structures
  in Thin Sections of Geological Samples with a Scanning Hall Effect
  Microscope. \emph{Sensors} \textbf{2019}, \emph{19}, 1636.

\bibitem{takahashi_ellipsoids_2017}
D.~Takahashi, V.~C. Oliveira~Jr. Ellipsoids (v1.0): 3-{D} magnetic modelling of
  ellipsoidal bodies. \emph{Geoscientific Model Development} \textbf{2017},
  \emph{10}, 3591--3608.

\bibitem{siqueira_fast_2017}
F.~C.~L. Siqueira, V.~C. Oliveira~Jr., V.~C.~F. Barbosa. Fast iterative
  equivalent-layer technique for gravity data processing: {A} method grounded
  on excess mass constraint. \emph{GEOPHYSICS} \textbf{2017}, \emph{82},
  G57--G69.

\bibitem{reis_estimating_2016}
A.~L.~A. Reis, V.~C. Oliveira~Jr., E.~Yokoyama, A.~C. Bruno, J.~M.~B. Pereira.
  Estimating the magnetization distribution within rectangular rock samples.
  \emph{Geochemistry, Geophysics, Geosystems} \textbf{2016}, \emph{17},
  3350--3374.

\bibitem{oliveira_jr_estimation_2015}
V.~Oliveira~Jr, D.~Sales, V.~Barbosa, L.~Uieda. Estimation of the total
  magnetization direction of approximately spherical bodies. \emph{Nonlinear
  Processes in Geophysics} \textbf{2015}, \emph{22}, 215--232.

\bibitem{uieda_geophysical_2014}
L.~Uieda, V.~C. Oliveira~Jr., V.~C.~F. Barbosa. Geophysical tutorial: {Euler}
  deconvolution of potential-field data. \emph{The Leading Edge} \textbf{2014},
  \emph{33}, 448--450.

\bibitem{melo_estimating_2013}
F.~F. Melo, V.~C.~F. Barbosa, L.~Uieda, V.~C. Oliveira~Jr., J.~B.~C. Silva.
  Estimating the nature and the horizontal and vertical positions of 3D
  magnetic sources using {Euler} deconvolution. \emph{GEOPHYSICS}
  \textbf{2013}, \emph{78}, J87--J98.

\bibitem{oliveira_3-d_2013}
V.~C. Oliveira~Jr., V.~C. Barbosa. 3-{D} radial gravity gradient inversion.
  \emph{Geophysical Journal International} \textbf{2013}, \emph{195}, 883--902.

\bibitem{oliveira_jr._polynomial_2013}
V.~C. Oliveira~Jr., V.~C.~F. Barbosa, L.~Uieda. Polynomial equivalent layer.
  \emph{GEOPHYSICS} \textbf{2013}, \emph{78}, G1--G13.

\bibitem{oliveira_jr_source_2011}
V.~C. Oliveira~Jr., V.~C.~F. Barbosa, J.~B.~C. Silva. Source geometry
  estimation using the mass excess criterion to constrain 3-{D} radial
  inversion of gravity data. \emph{Geophysical Journal International}
  \textbf{2011}, \emph{187}, 754--772.

\end{thebibliography}

